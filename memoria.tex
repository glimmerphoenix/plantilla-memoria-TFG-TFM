% !TeX TXS-program:compile = txs:///xelatex/[--shell-escape]
% !TeX TXS-program:bibliography = txs:///biber

%%%%%%%%%%%%%%%%%%%%%%%%%%%%%%%%%%%%%%%%%%%%%%%%%%%%%%%%%%%%%%%%%%%%%%%%%%%%%%%%
%% Plantilla de memoria en LaTeX para TFG/TFM - Universidad Rey Juan Carlos
%%
%% Por Gregorio Robles <grex arroba gsyc.urjc.es>
%%     Felipe Ortega   <felipe.ortega@urjc.es>
%%     Grupo de Sistemas y Comunicaciones (GSyC)
%%     Escuela Técnica Superior de Ingenieros de Telecomunicación
%%     Universidad Rey Juan Carlos
%%
%% (Muchas ideas tomadas de Internet, colegas del GSyC, antiguos alumnos...
%%  etc. Muchas gracias a todos)
%%
%% La última versión de esta plantilla está siempre disponible en:
%%     https://github.com/glimmerphoenix/plantilla-memoria-TFG-TFM
%%
%% - Ejecución en sistema local:
%% Para obtener el documento en PDF, se recomienda compilarlo con un editor
%% como TeXStudio (https://www.texstudio.org/) en las dos primeras líneas
%% de este archivo se incluyen ya las directrices para que el documento se
%% compile correctamente con esta herramienta.
%%
%% - %TODO: incluir un fichero makefile para facilitar la compilación
%%   en línea de comandos. 
%%
%% - Esta plantilla se compila utilizando el motor XeLaTeX. Es un compilador
%% de LaTeX más moderno que, entre otras mejoras, incluye soporte nativo para
%% codificación de caracteres Unicode (UTF-8), traducción políglota de referencias
%% (en combinación con  BibLaTex) y soporte para fuentes OpenTrueFonts (OTF)
%% instaladas previamente en el sistema. Esta última característic permite, por 
%% ejemplo, insertar iconos de la colección Fontawesome en el texto, o también
%% utilizar cualquiera de las tipografías de la colección Google Fonts.
%%
%% XeLaTeX viene ya incluido en todas las distribuciones modernas de LaTeX.
%%
%% - Todas las pruebas con este documento se han realizado empleando la distribución
%% de LaTeX TexLive (https://www.tug.org/texlive/)
%%
%% - IMPORTANTE: Esta plantilla usa XeLaTeX, BibLaTeX y el paquete minted, que
%% necesita el paquete python3-pygments instalado en tu sistema. También se pueden 
%% usar tipografías OTF o TrueType, siempre que estén instaladas en el sistema.
%%
%% Versión de TeX Live: 2024
%% Versión de LaTeX: LaTeX2e
%%
%% - Edición y ejecución en línea: 
%% Puedes descargar y subir la plantilla a una plataforma como Overleaf (https://www.overleaf.com/)
%% de edición colaborativa de LaTeX en línea. Overleaf ya tiene 
%% instalados todos los paquetes LaTeX y otras dependencias software para
%% que esta plantilla compile correctamente.
%%
%% NOTA 1: Si compilas este documento en Overleaf, recuerda cambiar
%% la configuración (botón "Menu" en la esquina superior izquierda de la interfaz)
%% y elegir la opción Compiler --> XeLaTeX. En caso contrario no funcionará.
%%
%% NOTA 2: Overleaf está restringiendo paulatinamente los recursos disponibles
%% para cuentas gratuitas. Es altamente recomendable usar una cuenta pro para
%% garantizar que los documentos más complejos compilen con éxito. 
%%
%% NOTA 3: las imágenes que incluuyas deben estar en formato PNG, JPG, EPS o PDF. También 
%% se pueden usar imágenes en otros formatos, pero se requieren algunos cambios en el 
%% preámbulo del documento.
%%
%% - Se aceptan peticiones de nuevas funcionalidades (feature requests) o contribuciones
%% (pull requests) a través del sitio del proyecto en GitHub.

%%%%%%%%%%%%%%%%%%%%%%%%%%%%%%%%%%%%%%%%%%%%%%%%%%%%%%%%%%%%%%%%%%%%%%%%%%%%%%%%

\RequirePackage[l2tabu,orthodox]{nag} % turn on warnings because of bad style

% We use scrbook document class, part of the KOMA-Script bundle
% Main link: https://ctan.org/pkg/koma-script
% Guía rápida (algo antigua): https://webs.um.es/jal/docs/05-Koma.pdf

% Quick tips for DIV global option values
% https://tex.stackexchange.com/questions/183172/koma-script-typearea-and-div
\documentclass[
paper=a4,
fontsize=12pt,
% Replace twoside with oneside if you are printing your doc in single-side
% paper or for viewing on screen.
%oneside,
twoside,
openright,                  % Always start new chapter on right page
titlepage,                  % Customize titlepage
headsepline,                % Add header separating line
%footsepline,               % Add footer separating line
%https://tex.stackexchange.com/questions/114570/koma-script-scrpage2-footer-height
footheight=16pt,            % Set up footer height
parskip=half+,				% Space between paragraphs: full, full+, half, half+
numbers=noenddot,			% Remove end dot in headers numbering
%chapterprefix=true,		% Include prefix for chapter headings
%appendixprefix=true,		% Include prefix for appendix headings
headinclude=true,           % Include page header
footinclude=true,           % Include page footer
cleardoublepage=empty,      % Use empty (blank) for page padding
captions=tableheading,      % Table captions are placed above -> calculate skip
BCOR=5mm, 				    % Set binding correction if necessary
%BCOR=0mm,                  % Uncomment for no biding extra margin (symmetric layout)
DIV=11    			        % =calc --> calculate optimal page layout;
% =x --> set x to a value in [6,15]
%mpinclude=true,			% width = textwidth
usegeometry=true			% Let typearea translate layout changes to geometry
]{scrbook}

%%--------------------------------------
% KOMA CLASSES
%%--------------------------------------
%
% The class "scrartcl" is one of the so-called KOMA-classes, a set of
% very well done LaTeX-classes that produce a very European layout
% (e.g. titles with a sans-serif font).
% 
% The KOMA classes have extensive documentation that you can access
% via the commands:
%   texdoc scrguide # in German
%   texdoc scrguien # in English
%   
%
% The available classes are:
%
% scrartcl - for "articles", typically for up to ~20 pages, the
%            highest level sectioning command is \section
%
% scrreprt - for "reports", typically for up to ~200 pages, the
%            highest level sectioning command is \chapter
%
% scrbook  - for "books", for more than 200 pages, the highest level
%            sectioning command is \part.
%
% USEFUL OPTIONS
%
% a4paper  - Use a4 paper instead of the default american letter
%            format.
%
% 11pt, 12pt, 10pt 
%          - Use a font with the given size.
%
% bibtotoc - Add the bibliography to the table of contents
%
% The KOMA-script classes have plenty of options to modify

%%--------------------------------------------------------------------

% Page layout and linespread

% Line and page breaking
\sloppy
\clubpenalty = 10000
\widowpenalty = 10000
\brokenpenalty = 10000

% At least +5% extra linespread for Palatino-like font
% Example: Tex Gyre Pagella is an open source Palatino-like font
\linespread{1.07}
% TODO: DOCUMENT THIS OPTION
%\KOMAoptions{DIV=last}
% Uncomment to adjust blank space between consecutive lines to one blank line
%%\setlength{\parskip}{\baselineskip}1\normalsize
% Uncomment to use advanced packages to manage indentation and blank spaces
%\usepackage[indent,parfill]{parskip}
\usepackage{ragged2e}				% Enhanced ragged commands

% --------------------------------------------------------------------
% GEOMETRY
% --------------------------------------------------------------------
% IMPORTANT: documentclass option usegeometry must be set to true
% to use the LaTeX package geometry for defining typesetting area dimensions

% Verbose option is for debuggin purposes
\usepackage[verbose]{geometry}

% Uncomment package for layout debugging (printing layout params values)
\usepackage{layouts}

% Default page layout with generous side margin
\RequirePackage{typearea}

% Uncomment to increase vertical text area in main body of page layout
%\setlength{\textheight}{1.2\textheight}
% Another option from KOMA-Script doc: \areaset[BCOR]{width}{height} macro

%%%-- Adjusting text area size
%%% NOTE: Increment the last param to increase vertical text area size

% From R. Bringhurst: prop. 1:1.6 (approx. 1:\phi) --> 82-86 chars./line w/ Tex Gyre Pagella font
\areaset[current]{\textwidth}{1.6\textwidth}

% From R. Bringhurst: prop. 1:1.414 (sqrt(2)) corresponds to A4 ISO standard.
% With Times New Roman (narrow font) this leads to long lines (80-85 chars./line)
% With Tex Gyre Pagella (more pleasant font, inspired by Palatino) this leads to 70-75 chars./line.
%\areaset[current]{\textwidth}{1.414\textwidth}

% Layout for Palatino font from Classic Thesis
% \areaset[current]{320pt}{732pt} % ~ 336 * factor 2 + 33 head + 42 \the\footskip 
% \areaset{336pt}{761pt} % 686 (factor 2.2) + 33 head + 42 head \the\footskip 10pt  

%%%-- Margin notes

% Adjust side margin width and separation
%\setlength{\marginparwidth}{4em}%
%\setlength{\marginparsep}{2em}%

% http://latexref.xyz/Marginal-notes.html

% Minimum vertical space between notes;            
% Separation marginpars by a line's worth of space.
\setlength\marginparpush{4pt}

% http://latexref.xyz/Marginal-notes.html
% Minimum vertical space between notes;            
% Separation marginpars by a line's worth of space.
%\setlength\marginparpush{4pt}

%%% ---- Geometry debugging

% DEBUGGING GEOMETRY, uncomment next line
%\geometry{showframe} % display margins for debugging page layout


% --------------------------------------------------------------------
% SIDENOTES
% --------------------------------------------------------------------

% Project marginal notes a third of the way into the normal margin
% cf. page 44 KOMA-Script manual. Use together with class option
% mpinclude=true
% \setlength{\marginparwidth}{1.67\marginparwidth}

% Try this package from KOMA-Script author for margin notes
%\usepackage{marginnote}

% More powerful margin notes
% Already loads package marginnote
\usepackage{sidenotes}
% Customize sidenotes font size --> \footnotesize
% Source: https://tex.stackexchange.com/questions/361622/changing-sidenote-size
\makeatletter
\RenewDocumentCommand\sidenotetext{ o o +m }{%      
    \IfNoValueOrEmptyTF{#1}{%
        \@sidenotes@placemarginal{#2}{\footnotesize\textsuperscript{\thesidenote}{}~\footnotesize#3}%
        \refstepcounter{sidenote}%
    }{%
        \@sidenotes@placemarginal{#2}{\textsuperscript{#1}~#3}%
    }%
}
\makeatother

% --------------------------------------------------------------------
% CAPTIONS
% --------------------------------------------------------------------

% Captions beside figures (located in the side margin)
% https://tex.stackexchange.com/questions/318151/koma-script-and-sidenotes-how-to-format-side-margin-caption-and-its-caption

\usepackage[hypcap=true]{caption}   	% Correctly placed anchors for hyperlinks
%\usepackage{floatrow}               	% Set up captions of floats
%x\usepackage{marginfix}             	% Make marginpars float freely
\usepackage{scrlayer-scrpage}       	% Customise head and foot regions
%\usepackage[footnote]{snotez}       	% Footnotes as sidenotes

% More floats
\extrafloats{100}

% --------------------------------------------------------------------
% TYPOGRAPHIES (FONTS)
% --------------------------------------------------------------------

% TODO: Lots of warnings produced by microtype, investigate why
% Keep the following line commented for now
%\usepackage[protrusion=true,final]{microtype} % Improve paragraph aspect in pdflatex

% Math fonts config
%% ------
% Source: https://www.tug.org/TUGboat/tb31-2/tb98robertson.pdf
%
% Use mathspec if most of your maths needs are fulfilled by a pre-existing maths package
% (such as mathpazo) but you would like your maths alphabets to be taken from the text font; 
% alternatively, use unicode-math if you have an OpenType maths font that you would like to use for typesetting all
% aspects of the mathematics.
%
%% ------

%\usepackage[intlimits]{amsmath}     % Provides all mathematical commands
\usepackage{amssymb}                % Math symbols
\usepackage{mathtools}              % More math symbols and environments; loads amsmath that should go before unicode-math
% TODO: CHECK ERRORS WITH MATHSPEC AND SETTING CUSTOM MATH FONT
% DISCOVERY: mathspec may have issues with recent versions of fontspec
%\usepackage{mathspec}

% Alternative package for setting math font
\usepackage[math-style=ISO]{unicode-math}

\usepackage{comment}
% Load the right font packages for XeLaTeX or LuaTeX
\usepackage{iftex}

\ifXeTeX
\usepackage[]{fontspec}
\else
\usepackage{luatextra}
\fi
%\defaultfontfeatures{Scale=MatchLowercase, Ligatures=TeX}

%% Set up main document fonts
%\setmainfont[]{TeX Gyre Pagella}	     % Similar to Palatino
\setmainfont[]{Times New Roman}
\setsansfont[]{Lato}                     % Sans-serif font, alt [Scale=0.9]
\setmonofont[]{Source Code Pro Medium}   % Monospaced font, alt [Scale=0.9]
% TODO: CHECK ERRORS WITH MATHSPEC AND SETTING CUSTOM MATH FONT
%% Math font
\setmathfont[]{TeX Gyre Pagella Math}    % Set math font

% Alternative: use package euler-math
% Euler Math is an OTF clone of Hermann Zapf and Donald Knuth’s Euler font
%
% 1. Loads unicode-math with the math-style=upright option and sets Euler-Math as maths font
%
% 2. Checks at \begin{document} if packages amssymb or latexsym are loaded and
%    issues warnings in case they are;
% 3. It provides aliases for glyphs named differently in Unicode, so that latexsym or
%    AMS names are also available;
% 4. It defines some specific math characters \varemptyset (∅), etc.

% -------------------------------------------
% Example: Here several variants of the Montserrat Google Font are configured
% for different part, chapter, section and subseciton headers
% You can use any other font that is installed in the system
% -------------------------------------------
% Custom fonts for headers and sections

\newfontfamily\headerfont{Montserrat}[Scale=MatchUppercase]
\newfontfamily\partfont{Montserrat}
\newfontfamily\chapterfont{Montserrat}
\newfontfamily\sectionfont{Montserrat-SemiBold}
\newfontfamily\subsectionfont{Montserrat-Medium}
\newfontfamily\subsubsectionfont{Montserrat-Medium}

% Use upquote if available, for straight quotes in verbatim environments
\IfFileExists{upquote.sty}{\usepackage{upquote}}{}

% Special fonts and glyphs
\usepackage{ccicons}                % Creative Commons icons
\usepackage{metalogo}               % XeTeX logo
\usepackage{fontawesome5}           % Fontawesome 5 icons
\usepackage{adforn}                 % Forns and glyphs

% --------------------------------------------------------------------
% LANGUAGE AND TRANSLATIONS
% --------------------------------------------------------------------

% Language translations in XeLaTeX
\usepackage{polyglossia}
%\setmainlanguage{english}          % When there is a single language
\setdefaultlanguage{spanish}        % Default lang in multilang doc
\setotherlanguages{english}         % Alt lang(s) in multilang doc
% Custom translations in polyglossia
% For additional options see the package manual 
\gappto\captionsspanish{%
\def\tablename{Tabla}% 
\def\listingscaption{Código}%
\def\refname{Referencias}%
\def\appendixname{Apéndice}%
\def\listtablename{Índice de tablas}%
\def\listoflistingsname{Índice de listas de código}
}

% --------------------------------------------------------------------
% GRAPHICS AND TABLES
% --------------------------------------------------------------------

% Load colour palettes
% Need to pass correct options for booktabs to load the package
\PassOptionsToPackage{dvipsnames,x11names,table}{xcolor}

% Nice rules for tables. Usage \begin{tabular}\toprule ... \midrule ... \bottomrule
\usepackage{booktabs}           % WARNING: this package already loads package xcolor
\usepackage{multirow}           % multiple row/column layout facilities
\usepackage{colortbl}           % For coloured tables
%\usepackage{subcaption}        % Adv. subcaption management
\usepackage{longtable}          % Tables spanning more than one page

%%-- Graphs and colors
%\usepackage{subfig}            % Subfigures
\usepackage{pgf}
\usepackage{graphicx}           % Embedding images
\setkeys{Gin}{width=\linewidth,totalheight=\textheight,keepaspectratio}
\graphicspath{{img/}}           % set paths for automated image search with \includegraphics{imagefile}

%%-- Customized GSyC colours (taken from logo)
\definecolor{GSyCblue}{RGB}{18,46,116} % Dark blue from GSyC logo (top part of fonts)
\definecolor{GSyClightblue}{RGB}{74,146,212} % Light blue from GSyC logo (low part of "y")

% --------------------------------------------------------------------
% SECTION HEADINGS
% --------------------------------------------------------------------

% Here, we configure customized colours for different section headers

%%-- Customize heading style
%%%-- Custom part heading
%\RedeclareSectionCommand[style=part,
%                         beforeskip=0pt,
%                         afterskip=6ex]{part}    	
\renewcommand*\partformat{\partfont\color{GSyCblue}\scshape Parte \thepart\autodot\enskip}	% Skip \partname in part heading
%%-- Configure fonts and colors for all other headings
\addtokomafont{part}{\centering\partfont\color{GSyCblue}\scshape}
\addtokomafont{chapter}{\chapterfont\color{GSyCblue}}
\addtokomafont{section}{\normalfont\sectionfont\color{GSyCblue}}
\addtokomafont{subsection}{\normalfont\subsectionfont\color{GSyCblue}}
\addtokomafont{subsubsection}{\normalfont\subsubsectionfont\color{GSyCblue}}

\newcommand\titlerule[1][1pt]{\rule{\textwidth}{#1}}
\renewcommand\partlineswithprefixformat[3]{%
    \Ifstr{#2}{}{}{\color{GSyCblue}\titlerule[1.5pt]\vspace{\baselineskip}\par\nobreak}%
    #2#3\par\nobreak\vspace{\baselineskip}\titlerule[3pt]%
}

% --------------------------------------------------------------------
% TABLE OF CONTENTS
% --------------------------------------------------------------------

% Adjust indentation in ToC (KOMA-Script style)

% \RedeclareSectionCommands[tocdynnumwidth]{chapter,section,subsection}
\RedeclareSectionCommand[
tocnumwidth+=-6pt
]{part}

% Set space between chapter entry and rest of chapter contents
\DeclareTOCStyleEntries[
onstarthigherlevel= \vspace{0.5\baselineskip},
%onstartlowerlevel= \vspace{\baselineskip}
]{default}{section} % <- Higher level in report and book is chapter

% Indent chapters and rest of sections to the right of part entries
%\RedeclareSectionCommand[tocnumwidth=3.5em]{chapter}
%\renewcommand\addchaptertocentry[2]{%
    %	\Ifstr{#1}{}{%
        %		\addtocentrydefault{chapter}{#1}{#2}%
        %	}{%
        %		\addtocentrydefault{chapter}{\hspace*{2em}#1}{#2}%
        %}}

% Prepend 'partname' to part entry in ToC
\DeclareTOCStyleEntry[
entrynumberformat=\entrynumberwithprefix{\partname},
dynnumwidth
]{tocline}{part}
\newcommand\entrynumberwithprefix[2]{#1\enspace#2\hfill}

%% --- Even more examples

%% Prepend 'chaptername' to chapter entry in ToC
%\usepackage{tocbasic}
%\DeclareTOCStyleEntry[
%entrynumberformat=\entrynumberwithprefix{\chaptername},
%dynnumwidth
%]{tocline}{chapter}
%
%% Fix appendices prefix
%% https://tex.stackexchange.com/questions/515515/koma-script-write-appendix-in-the-table-of-contents
%\newcommand\useprefix[2]{#1#2}
%\newlength\appendixprefixwidth
%
%\NewDocumentCommand\appendixprefixintoc{}
%{%
%	\setlength\appendixprefixwidth{%
%		\widthof{\usekomafont{chapterentry}\appendixname~}}% measure needed additional space
%	\DeclareTOCStyleEntry
%	[%
%	entrynumberformat=\useprefix{\appendixname~},% add the prefix before the entrynumber
%	numwidth+=\appendixprefixwidth% enlarge numwidth for level section
%	]{default}{chapter}
%	\DeclareTOCStyleEntries[%
%	indent+=\appendixprefixwidth% enlarge indent for other levels
%	]{default}{section,subsection,subsubsection,paragraph,subparagraph}
%}
%
%\newcommand*{\appendixmore}{%
%	\renewcommand*{\sectionformat}{%
%		\appendixname~\thesection\autodot\enskip}%
%	\renewcommand*{\sectionmarkformat}{%
%		\appendixname~\thesection\autodot\enskip}%
%	\addtocontents{toc}{\appendixprefixintoc}%
%}

% Uncomment the following lines if prefixes for chapters and appendices are included
% Adjust section and subsection identation to compesate prefixes
%\RedeclareSectionCommand[tocindent=5.3em]{section}
%\RedeclareSectionCommand[tocindent=7.7em]{subsection}

% --------------------------------------------------------------------
% BIBLIOGRAPHY (Using BibLaTeX)
% --------------------------------------------------------------------

% https://www.overleaf.com/learn/latex/Biblatex_bibliography_styles
% https://www.overleaf.com/learn/latex/biblatex_citation_styles

\usepackage[
backend=biber,
style=numeric,
sorting=nty
]{biblatex}
\addbibresource{memoria.bib}
\DeclareFieldFormat{url}{\mkbibacro{URL}\addcolon\nobreakspace\url{#1}}

% --------------------------------------------------------------------
% FORMATTING STEM CONTENTS
% --------------------------------------------------------------------

% units package lets you set and display units
\usepackage[ugly]{units}        % Allows you to type units with correct spacing and font style. 
% Usage: $\unit[100]{m}$ or $\unitfrac[100]{m}{s}$

%\usepackage{grffile}           % Allow including some images (like graphicx). Usage: \includegraphics{path/to/file}

% Math environments
% Aquí definimos las etiquetas que se mostrarán al invocar cada entorno matemático
\newtheorem{theorem}{Teorema}
\newtheorem{corollary}[theorem]{Corolario}
\newtheorem{lemma}[theorem]{Lema}
\newtheorem{definition}[theorem]{Definición}

% Se pueden definir más etiquetas, en inglés u otros idiomas en el mismo documento.

% --------------------------------------------------------------------
% ADDITIONAL PACKAGES AND HACKS
% --------------------------------------------------------------------

\usepackage{url}                % Lets you typeset urls. Usage: \url{http://...}

% Use \xpsace in macros to automatically insert space based on context. 
% Usage: \newcommand{\es}{ESPResSo\xspace}
\usepackage{xspace}             

%\usepackage{epigraph}           % Typesetting relevant quotations after section heading

%%%-- Some useful packages for adding test content

% Blindtext
% Opciones pangram, bible, random (defecto)
\usepackage[pangram]{blindtext}
% Lorem ipsum
\usepackage{lipsum}
\usepackage{kantlipsum}

%%%-- Verbatim environments

\usepackage{fancyvrb}           % extended verbatim environments
\fvset{fontsize=\normalsize}    % default font size for fancy-verbatim environments

%%%-- Repositioning margin figures

% Trick to place margin figures in correct page
\usepackage{mparhack}
% \usepackage{marginfix}		% Explore this package to fix probs. with marginotes

% Set lists (itemize, enumerate) with customized icons
% easy restart of number sequence in enumerate
% http://ctan.org/pkg/enumitem
\usepackage[shortlabels]{enumitem}

\newcommand{\usageitem}[1]{%
    \item[%
    {\makebox[2em]{\strut\color{GSyCblue} #1}}%
    ]
}

% --------------------------------------------------------------------
% COLOURED BOXES
% --------------------------------------------------------------------

% Colored boxes for information corners
\usepackage{tcolorbox}% http://ctan.org/pkg/tcolorbox

% Caja azul con título para información, explicaciones extra y detalles
% Icons: \faInfoCircle \faBook \faBookmark
\newtcolorbox{mybluebox}[1]{colback=darkblue!5!white,colframe=darkblue!75!black,
fonttitle=\normalfont,title=#1}

% Caja verde para comentarios adicionales
% Icons: \faComment \faCrosshairs
\newtcolorbox{mygreenbox}[1]{colback=darkgreen!5!white,colframe=darkgreen!75!black,
fonttitle=\normalfont,title=#1}

% TODO: Yellow box with title for warnings

% Caja roja con título para posibles problemas explicaciones extra y detalles
% Icons: \faExclamationCircle \faBomb
\newtcolorbox{myredbox}[1]{colback=darkred!5!white,colframe=darkred!75!black,
fonttitle=\normalfont,title=#1}

% Coloured equations

\usepackage{empheq}

% --------------------------------------------------------------------
% SYNTAX HIGHLIGHTING
% --------------------------------------------------------------------

% Doc: https://ctan.org/pkg/minted
\usepackage[chapter]{minted}

% Add here custom short commands to colour inline code
% TODO: INSERT AN EXAMPLE OF THIS USEFUL FEATURE

% Source code listings
\usepackage{listings}           % Source Code Listings. Usage: \begin{lstlisting}...\end{lstlisting}
\lstset{
    basicstyle=\ttfamily,
    columns=fullflexible,
    frame=single,
    breaklines=true,
    postbreak=\mbox{\textcolor{red}{$\hookrightarrow$}\space},
}

\lstloadlanguages{python}       % Default highlighting set to "python"
% Fix issue with internal hacks in listings using deprecated KOMA-Script commands
% See: https://tex.stackexchange.com/questions/51867/koma-warning-about-toc
\usepackage[listings=false]{scrhack}

\usepackage{csquotes} % Para traducciones con biblatex

% --------------------------------------------------------------------
% GLOSSARY
% --------------------------------------------------------------------

\usepackage[acronym]{glossaries}
\makeglossaries
\loadglsentries{glossary}

% --------------------------------------------------------------------
% LINKS
% --------------------------------------------------------------------

% hyperref package is loaded at the end of the preamble to avoid conflicts
\usepackage{hyperref}  	        % Provides clickable links in the PDF-document for \ref

% --------------------------------------------------------------------
% DOC METADATA
% --------------------------------------------------------------------

\title{Memoria del Proyecto}
\author{Nombre del autor}

%% Config hyperlink options

\hypersetup{
    pdftoolbar=true,	% Muestra barra de herramientas en Adobe Acrobat
	pdfmenubar=true,	% Muestra menú en Adobe Acrobat
	pdftitle={Título del TFG/TFM},
	pdfauthor={Nombre del alumno/a},
	pdfcreator={EIF, URJC},
	pdfproducer={XeLaTeX},
	pdfsubject={Topic1, Topic2, Topic3},
	pdfnewwindow=true,              %links open in new window
    colorlinks=true,                % false: boxed links; true: coloured links
    linkcolor=Firebrick4,           % enlaces internos 
    citecolor=Aquamarine4,          % enlaces a citas bibliográficas
    urlcolor=RoyalBlue3,            % hiperenlances ordinarios
    linktocpage=true                % Enlaces en núm. pág. en ToC
}

%%%---------------------------------------------------------------------------
% Comentarios en línea de revisión
% Este bloque se puede borrar cuando finalizamos el borrador

% \usepackage[colorinlistoftodos]{todonotes}
% \usepackage{verbatim}
%%%---------------------------------------------------------------------------

\begin{document}

%---------------------------------------------------------------------

%---------------------------------------------------------------------

% KOMA-Script organization
% Frontmatter (roman page numbering, among other default features)
\frontmatter

%---------------------------------------------------------------------

%%--------------------------------------------------------------------

% Declare new geometry for frontmatter if needed
% IMPORTANT: Requires \usepackage{geometry}; invalidates DIV global option
\newgeometry{margin=2.5cm}

% --------------------------------------------------------------------
% COVER PAGE
% --------------------------------------------------------------------
 

%-------------------------------------------------------------------------------
%%%--- PORTADA

\begin{titlepage}
\begin{center}

\includegraphics[width=0.4\textwidth]{img/LogoURJC.png}

\vspace{0.5cm}

\Large 
ESCUELA DE INGENIERÍA DE FUENLABRADA

\vspace{1.5cm}

\Large 
NOMBRE DE LA TITULACIÓN EN MAYÚSCULAS

\vspace{1.5cm}

\Large
\textbf{TRABAJO FIN DE GRADO}

\vspace{2cm}

\large UN TÍTULO DE PROYECTO LARGO\\
EN DOS LÍNEAS

\vspace{1.5cm}

\large
Autor/a : Nombre del Alumno/a \\
\vspace{0.5cm}
Tutor/a : Dr. Nombre del Profesor/a

\vspace{3cm}

\large
Curso Académico 20XX/20XX


\end{center}
\end{titlepage}

\newpage
\mbox{}
\thispagestyle{empty} % para que no se numere esta pagina


%-------------------------------------------------------------------------------
%%%--- Página de firmas (por motivos históricos)
\clearpage
\pagenumbering{gobble}
\chapter*{}

\vspace{-4cm}
\begin{center}
\LARGE
\textbf{Trabajo Fin de Grado/Máster}

\vspace{1cm}
\large
Título del Trabajo con Letras Capitales para Sustantivos y Adjetivos

\vspace{1cm}
\large
\textbf{Autor/a :} Nombre del Alumno/a  \\
\textbf{Tutor/a :} Dr. Nombre del profesor/a

\end{center}

\vspace{1cm}
La defensa del presente Proyecto Fin de Grado/Máster se realizó el día 3\qquad$\;\,$ de
\qquad\qquad\qquad\qquad \newline de 20XX, siendo calificada por el siguiente tribunal:


\vspace{0.5cm}
\textbf{Presidente:}

\vspace{0.8cm}
\textbf{Secretario:}

\vspace{0.8cm}
\textbf{Vocal:}


\vspace{0.8cm}
y habiendo obtenido la siguiente calificación:

\vspace{0.8cm}
\textbf{Calificación:}


\vspace{0.8cm}
\begin{flushright}
Móstoles/Fuenlabrada, a \qquad$\;\,$ de \qquad\qquad\qquad\qquad de 20XX
\end{flushright}

%-------------------------------------------------------------------------------
%%%--- Dedicatoria

\chapter*{}
%\pagenumbering{Roman} % para comenzar la numeración de paginas en numeros romanos
\begin{flushright}
\textit{Aquí normalmente \\
se inserta una dedicatoria corta \\}
\end{flushright}

%-------------------------------------------------------------------------------
%%%--- Agradecimientos

\chapter*{Agradecimientos}
%\addcontentsline{toc}{chapter}{Agradecimientos} % si queremos que aparezca en el índice
\markboth{AGRADECIMIENTOS}{AGRADECIMIENTOS} % encabezado 

Aquí vienen los agradecimientos\ldots

Hay más espacio para explayarse y explicar a quién agradeces su apoyo o ayuda para
haber acabado el proyecto: familia, pareja, amigos, compañeros de clase\ldots

También hay quien, en algunos casos, hasta agradecer a su tutor o tutores del proyecto
la ayuda prestada\ldots

%-------------------------------------------------------------------------------
%%%--- Resumen

\chapter*{Resumen}
%\addcontentsline{toc}{chapter}{Resumen} % si queremos que aparezca en el índice
\markboth{RESUMEN}{RESUMEN} % encabezado

Aquí viene un resumen del proyecto.
Ha de constar de tres o cuatro párrafos, donde se presente de manera clara y concisa de qué va el proyecto. 
Han de quedar respondidas las siguientes preguntas:

\begin{itemize}
  \item ¿De qué va este proyecto? ¿Cuál es su objetivo principal?
  \item ¿Cómo se ha realizado? ¿Qué tecnologías están involucradas?
  \item ¿En qué contexto se ha realizado el proyecto? ¿Es un proyecto dentro de un marco general?
\end{itemize}

Lo mejor es escribir el resumen al final.

%-------------------------------------------------------------------------------
%%%--- Resumen en inglés

\chapter*{Summary}
%\addcontentsline{toc}{chapter}{Summary} % si queremos que aparezca en el índice
\markboth{SUMMARY}{SUMMARY} % encabezado

Here comes a translation of the ``Resumen'' into English. 
Please, double check it for correct grammar and spelling.
As it is the translation of the ``Resumen'', which is supposed to be written at the end, this as well should be filled out just before submitting.

%-------------------------------------------------------------------------------
% Lista de comentarios de revisión
% Se puede borrar este bloque al acabar el borrador

%\listoftodos
%\markboth{TODO LIST}{TODO LIST} % encabezado
%-------------------------------------------------------------------------------

% Ends custom geometry for the frontmatter if it was changed
\restoregeometry

%%%%%%%%%%%%%%%%%%%%%%%%%%%%%%%%%%%%%%%%%%%%%%%%%%%%%%%%%%%%%%%%%%%%%%%%%%%%%%%%
%%%%%%%%%%%%%%%%%%%%%%%%%%%%%%%%%%%%%%%%%%%%%%%%%%%%%%%%%%%%%%%%%%%%%%%%%%%%%%%%
% ToC AND LISTS %
%-------------------------------------------------------------------------------

% Las buenas noticias es que los índices se generan automáticamente.
% Lo único que tienes que hacer es elegir cuáles quieren que se generen,
% y comentar/descomentar esa instrucción de LaTeX.

%%-- Índice de contenidos
\tableofcontents 

%%-- Índice de figuras
%\addcontentsline{toc}{chapter}{Lista de figuras} % para que aparezca en el indice de contenidos
\cleardoublepage
\listoffigures % indice de figuras

%%-- Índice de tablas
%\addcontentsline{toc}{chapter}{Lista de tablas} % para que aparezca en el indice de contenidos
\cleardoublepage
\listoftables % indice de tablas

%%-- Índice de fragmentos de código
\cleardoublepage
\listoflistings

%-------------------------------------------------------------------------------

%-------------------------------------------------------------------------------
% KOMA-Script organization
% Mainmatter (arabic page numbering, among other default features)
\mainmatter
%-------------------------------------------------------------------------------

%--- Begin main content

%-------------------------------------------------------------------------------

%-------------------------------------------------------------------------------
% INTRODUCCIÓN %
%-------------------------------------------------------------------------------

\cleardoublepage
\chapter{Introducción}
\label{sec:intro}
\pagenumbering{arabic} % para empezar la numeración de página con números

En este capítulo se introduce el proyecto.
Debería tener información general sobre el mismo, dando la información sobre el contexto en el que se ha desarrollado.

No te olvides de echarle un ojo a la página con los cinco errores de escritura más frecuentes\footnote{\url{http://www.tallerdeescritores.com/errores-de-escritura-frecuentes}}.

Aconsejo a todo el mundo que mire y se inspire en memorias pasadas.
Las memorias de los proyectos que he llevado yo están (casi) todas almacenadas en mi web del GSyC\footnote{\url{https://gsyc.urjc.es/~grex/pfcs/}}.

\section{Sección}
\label{sec:seccion}

Esto es una sección, que es una estructura menor que un capítulo. 

Por cierto, a veces me comentáis que no os compila por las tildes.
Eso es un problema de codificación.
Al guardar el archivo, guardad la codificación de ``ISO-Latin-1'' a ``UTF-8'' (o viceversa) y funcionará.

\subsection{Estilo}
\label{subsec:estilo}

Recomiendo leer los consejos prácticos sobre escribir documentos científicos en \LaTeX \ de Diomidis Spinellis\footnote{\url{https://github.com/dspinellis/latex-advice}}.

Lee sobre el uso de las comas\footnote{\url{http://narrativabreve.com/2015/02/opiniones-de-un-corrector-de-estilo-11-recetas-para-escribir-correctamente-la-coma.html}}. 
Las comas en español no se ponen al tuntún.
Y nunca, nunca entre el sujeto y el predicado (p.ej. en ``Yo, hago el TFG'' sobre la coma).
La coma no debe separar el sujeto del predicado en una oración, pues se cortaría la secuencia natural del discurso.
No se considera apropiado el uso de la llamada coma respiratoria o \emph{coma criminal}.
Solamente se suele escribir una coma para marcar el lugar que queda cuando omitimos el verbo de una oración, pero es un caso que se da de manera muy infrecuente al escribir un texto científico (p.ej. ``El Real Madrid, campeón de Europa'').

A continuación, viene una figura, la Figura~\ref{figura:foro_hilos}. 
Observarás que el texto dentro de la referencia es el identificador de la figura (que se corresponden con el ``label'' dentro de la misma). 
También habrás tomado nota de cómo se ponen las ``comillas dobles'' para que se muestren correctamente. 
Nota que hay unas comillas de inicio (``) y otras de cierre (''), y que son diferentes.
Volviendo a las referencias, nota que al compilar, la primera vez se crea un diccionario con las referencias, y en la segunda compilación se ``rellenan'' estas referencias. 
Por eso hay que compilar dos veces tu memoria.
Si no, no se crearán las referencias.

\begin{figure}
    \centering
    \includegraphics[bb=0 0 800 600, width=12cm, keepaspectratio]{img/foro1}
    \caption{Página con enlaces a hilos}
    \label{figura:foro_hilos}
\end{figure}

A continuación un bloque ``verbatim'', que se utiliza para mostrar texto tal cual.
Se puede utilizar para ofrecer el contenido de correos electrónicos, código, entre otras cosas.

{\footnotesize
    \begin{verbatim}
        From gaurav at gold-solutions.co.uk  Fri Jan 14 14:51:11 2005
        From: gaurav at gold-solutions.co.uk (gaurav_gold)
        Date: Fri Jan 14 19:25:51 2005
        Subject: [Mailman-Users] mailman issues
        Message-ID: <003c01c4fa40$1d99b4c0$94592252@gaurav7klgnyif>
        Dear Sir/Madam,
        How can people reply to the mailing list?  How do i turn off
        this feature? How can i also enable a feature where if someone
        replies the newsletter the email gets deleted?
        Thanks
        From msapiro at value.net  Fri Jan 14 19:48:51 2005
        From: msapiro at value.net (Mark Sapiro)
        Date: Fri Jan 14 19:49:04 2005
        Subject: [Mailman-Users] mailman issues
        In-Reply-To: <003c01c4fa40$1d99b4c0$94592252@gaurav7klgnyif>
        Message-ID: <PC173020050114104851057801b04d55@msapiro>
        gaurav_gold wrote:
        >How can people reply to the mailing list?  How do i turn off
        this feature? How can i also enable a feature where if someone
        replies the newsletter the email gets deleted?
        See the FAQ
        >Mailman FAQ: http://www.python.org/cgi-bin/faqw-mm.py
        article 3.11
    \end{verbatim}
}

%%-- Objetivos del  proyecto
%%-- Si la sección anterior ha quedado muy extensa, se puede considerar convertir
%%-- Las siguientes tres secciones en un capítulo independiente de la memoria

\section{Objetivos del proyecto}
\label{sec:objetivos}

\subsection{Objetivo general} % título de subsección (se muestra)
\label{sec:objetivo-general} % identificador de subsección (no se muestra, es para poder referenciarla)


Aquí vendría el objetivo general en una frase:
Mi Trabajo Fin de Grado/Master consiste en crear de una herramienta de análisis de los comentarios jocosos en repositorios de software libre alojados en la plataforma GitHub.

Recuerda que los objetivos siempre vienen en infinitivo.


\subsection{Objetivos específicos}
\label{sec:objetivos-especificos}

Los objetivos específicos se pueden entender como las tareas en las que se ha desglosado el objetivo general. Y, sí, también vienen en infinitivo.

Lo mejor suele ser utilizar una lista no numerada, como sigue:

\begin{itemize}
    \item Un objetivo específico.
    \item Otro objetivo específico.
    \item Tercer objetivo específico.
    \item \ldots
\end{itemize}

\section{Planificación temporal}
\label{sec:planificacion-temporal}

Es conveniente que incluyas una descripción de lo que te ha llevado realizar el trabajo.
Hay gente que añade un diagrama de GANTT.
Lo importante es que quede claro cuánto tiempo has consumido en realizar el TFG/TFM 
(tiempo natural, p.ej., 6 meses) y a qué nivel de esfuerzo (p.ej., principalmente los 
fines de semana).

\section{Estructura de la memoria}
\label{sec:estructura}

Por último, en esta sección se introduce a alto nivel la organización del resto del documento
y qué contenidos se van a encontrar en cada capítulo.

\begin{itemize}
    \item En el primer capítulo se hace una breve introducción al proyecto, se describen los objetivos del mismo y se refleja la planificación temporal.
    \item En el siguiente capítulo se describen las tecnologías utilizadas en el desarrollo de este TFM/TFG (Capítulo~\ref{chap:tecnologias}).
    \item En el capítulo~\ref{chap:diseño} Se describe el proceso de desarrollo
    de la herramienta \ldots
    \item En el capítulo~\ref{chap:experimentos} Se presentan las principales pruebas realizadas
    para validación de la plataforma/herramienta\ldots (o resultados de los experimentos
    efectuados).
    \item Por último, se presentan las conclusiones del proyecto así como los trabajos futuros que podrían derivarse de éste (Capítulo~\ref{chap:conclusiones}).
\end{itemize}


%-------------------------------------------------------------------------------
% ESTADO DEL ARTE %
%-------------------------------------------------------------------------------

\cleardoublepage
\chapter{Estado del arte}               %% a.k.a "Tecnologías utilizadas"
\label{chap:tecnologias}

Descripción de las tecnologías que utilizas en tu trabajo. 
Con dos o tres párrafos por cada tecnología, vale. 
Se supone que aquí viene todo lo que no has hecho tú.

Puedes citar libros, como el de Bonabeau et al., sobre procesos estigmérgicos~\cite{bonabeau:_swarm}. 
Me encantan los procesos estigmérgicos.
Deberías leer más sobre ellos.
Pero quizás no ahora, que tenemos que terminar la memoria para sacarnos por fin el título.
Nota que el \~ \ añade un espacio en blanco, pero no deja que exista un salto de línea. 
Imprescindible ponerlo para las citas.

Citar es importantísimo en textos científico-técnicos. 
Porque no partimos de cero.
Es más, partir de cero es de tontos; lo suyo es aprovecharse de lo ya existente para construir encima y hacer cosas más sofisticadas.
¿Dónde puedo encontrar textos científicos que referenciar?
Un buen sitio es Google Scholar\footnote{\url{http://scholar.google.com}}.
Por ejemplo, si buscas por ``stigmergy libre software'' para encontrar trabajo sobre software libre y el concepto de \emph{estigmergia} (¿te he comentado que me gusta el concepto de estigmergia ya?), encontrarás un artículo que escribí hace tiempo cuyo título es ``Self-organized development in libre software: a model based on the stigmergy concept''.
Si pulsas sobre las comillas dobles (entre la estrella y el ``citado por ...'', justo debajo del extracto del resumen del artículo, te saldrá una ventana emergente con cómo citar.
Abajo a la derecha, aparece un enlace BibTeX.
Púlsalo y encontrarás la referencia en formato BibTeX, tal que así:

\clearpage
{\footnotesize
    \begin{minted}{bibtex}
        @inproceedings{robles2005self,
            title={Self-organized development in libre software:
                a model based on the stigmergy concept},
            author={Robles, Gregorio and Merelo, Juan Juli\'an 
                and Gonz\'alez-Barahona, Jes\'us M.},
            booktitle={ProSim'05},
            year={2005}
        }
    \end{minted}
}

Copia el texto en BibTeX y pégalo en el fichero \texttt{memoria.bib}, que es donde están las referencias bibliográficas.
Para incluir la referencia en el texto de la memoria, deberás citarlo, como hemos hecho antes con~\cite{bonabeau:_swarm}, lo que pasa es que en vez de el identificador de la cita anterior (bonabeau:\_swarm), tendrás que poner el nuevo (robles2005self).
Compila el fichero \texttt{memoria.tex} (\texttt{pdflatex memoria.tex}), añade la bibliografía (\texttt{bibtex memoria.aux}) y vuelve a compilar \texttt{memoria.tex} (\texttt{pdflatex memoria.tex})\ldots y \emph{voilà} ¡tenemos una nueva cita~\cite{robles2005self}!

También existe la posibilidad de poner notas al pie de página, por ejemplo, una para indicarte que visite la página del GSyC\footnote{\url{http://gsyc.es}}.

\section{Sección 1} 
\label{sec:seccion1}

Hemos hablado de cómo incluir figuras, pero no se ha descrito cómo incluir tablas.
A continuación se presenta un ejemplo de tabla, la Tabla \ref{tabla:ejemplo} (fíjate 
en cómo se introduce una referencia a la tabla).

\begin{table}[htb]
    \begin{center}
        \caption{Ejemplo de tabla. Aquí viene una pequeña descripción (el \emph{caption}) del contenido de la tabla. Si la tabla no es autoexplicativa, siempre viene bien aclararla aquí.}
        \label{tabla:ejemplo}
        \begin{tabular}{ | l | c | r |} % tenemos tres colummnas, la primera alineada a la izquierda (l), la segunda al centro (c) y la tercera a la derecha (r). El | indica que entre las columnas habrá una línea separadora.
            \hline
            Uno & 2 & 3 \\ \hline % el hline nos da una línea vertical
            Cuatro & 5 & 6 \\ \hline
            Siete & 8 & 9 \\
            \hline
        \end{tabular}
    \end{center}
\end{table}

La Tabla \ref{tabla:ejemplo} utiliza un formato muy básico. En documentos más recientes, se suele utilizar
un diseño de tabla con un estilo más legible, utilizando el paquete \texttt{booktabs} que se añade en el preámbulo
del documento principal con el comando \mintinline{latex}|\usepackage{booktabs}|. Entre otros elementos de
interés, este paquete evita el uso de líneas verticales para separar las columnas y define tres tipos de líneas 
horizontales para construir nuestra tabla:

\begin{itemize}
    \item La línea superior, que delimita el comienzo de la tabla, con el comando \mintinline{latex}|toprule|.
    
    \item La línea intermedia, que separa el encabezado del contenido de la tabla, o diferentes secciones de
    contenido en ésta. Se inserta con el comando \mintinline{latex}|midrule|.
    
    \item La línea inferior, que marca el final de la tabla, con el comando \mintinline{latex}|bottomrule|.
\end{itemize}

Veamos un ejemplo de este nuevo estilo, que se muestra en la Tabla \ref{tabla:ejemplo-booktabs}. Además, también
se ilustra cómo emplear el paquete \texttt{multirow}, para gestionar filas o columnas que se expanden en múltiples
celdas de la parrilla de diseño de la tabla.

% Source: https://nhigham.com/2019/11/19/better-latex-tables-with-booktabs/

\begin{table}[hb]
    \centering
    \caption{Segundo ejemplo de tabla, esta vez usando el estilo que define el paquete \texttt{booktabs}.}
    \label{tbl:table-id}
    % The @{} symbol reduce the space at the edges of some of the widest tables 
    % by shortening the horizontal rules
    
    \begin{tabular}{@{}lcccccl@{}}\toprule
        & \multicolumn{3}{c}{$\mathrm{tol}=u_{single}$} & \multicolumn{3}{c}{$\mathrm{tol}=u_{double}$}
        \\\cmidrule(lr){2-4}\cmidrule(lr){5-7}
        & $mv$  & Rel.~err & Time    & $mv$  & Rel.~err & Time\\
        \midrule
        trigmv    & 11034 & 1.3e-7 & 3.9 & 15846 & 2.7e-11 & 5.6 \\
        trigexpmv & 21952 & 1.3e-7 & 6.2 & 31516 & 2.7e-11 & 8.8 \\
        trigblock & 15883 & 5.2e-8 & 7.1 & 32023 & 1.1e-11 & 1.4e1\\
        expleja   & 11180 & 8.0e-9 & 4.3 & 17348 & 1.5e-11 & 6.6 \\
        \bottomrule
    \end{tabular}
\end{table}

\section{Entorno de desarrollo: PyCharm}
\label{sec:entorno_de_desarrollo}

%%-- El comando \gls{} permite incluir términos en el glosario, para luego reunirlos todos
%%-- en una tabla al comienzo o al final del documento, junto con sus definiciones.

PyCharm es un \gls{ide} dedicado concretamente a la programación en Python y desarrollado por la compañía checa JetBrains.

Proporciona análisis de código, un depurador gráfico, una consola de Python integrada, control de versiones y, además, soporta desarrollo web con Django. Todas estas características lo convierten en un entorno completo e intuitivo, idóneo para el desarrollo de proyectos académicos como el que nos ocupa.


\section{Redacción de la memoria: LaTeX/Overleaf}
\label{sec:redaccion_de_la_memoria}

LaTeX es un sistema de composición tipográfica de alta calidad que incluye características especialmente diseñadas para la producción de documentación técnica y científica. Estas características, entre las que se encuentran la posibilidad de incluir expresiones matemáticas, fragmentos de código, tablas y referencias, junto con el hecho de que se distribuya como software libre, han hecho que LaTeX se convierta en el estándar de facto para la redacción y publicación de artículos académicos, tesis y todo tipo de documentos científico-técnicos. 

Por su parte, Overleaf es un editor LaTeX colaborativo basado en la nube. Lanzado originalmente en 2012, fue creado por dos matemáticos que se inspiraron en su propia experiencia en el ámbito académico para crear una solución satisfactoria para la escritura científica colaborativa.

Además de por su perfil colaborativo, Overleaf destaca porque, pese a que en LaTeX el escritor utiliza texto plano en lugar de texto formateado (como ocurre en otros procesadores de texto como Microsoft Word, LibreOffice Writer y Apple Pages), éste puede visualizar en todo momento y paralelamente el texto formateado que resulta de la escritura del código fuente.


%-------------------------------------------------------------------------------
% DISEÑO E IMPLEMENTACIÓN %
%-------------------------------------------------------------------------------

\cleardoublepage
\chapter{Diseño e implementación}
\label{chap:diseño}


Aquí viene todo lo que has hecho tú (tecnológicamente). 
Puedes entrar hasta el detalle. 
Es la parte más importante de la memoria, porque describe lo que has hecho tú.
Eso sí, normalmente aconsejo no poner código, sino diagramas.

\section{Arquitectura general} 
\label{sec:arquitectura}

Si tu proyecto es un software, siempre es bueno poner la arquitectura (que es cómo se estructura tu programa a ``vista de pájaro'').

Por ejemplo, puedes verlo en la Figura~\ref{fig:arquitectura}.
\LaTeX \ pone las figuras donde mejor cuadran. 
Y eso quiere decir que quizás no lo haga donde lo hemos puesto\ldots
Eso no es malo.
A veces queda un poco raro, pero es la filosofía de \LaTeX: tú al contenido, que yo me encargo de la maquetación.

\begin{figure}
    \centering
    \includegraphics[width=9cm, keepaspectratio]{img/arquitectura.png}
    \caption{Estructura del parser básico.}\label{fig:arquitectura}
\end{figure}

\begin{figure}
    \centering
    \includegraphics[bb=0 0 800 600, width=12cm, keepaspectratio]{img/foro1}
    \caption{Página con enlaces a hilos}\label{fig:_arquitectura}
\end{figure}


Recuerda que toda figura que añadas a tu memoria debe ser explicada.
Sí, aunque te parezca evidente lo que se ve en la Figura~\ref{fig:arquitectura}, la figura en sí solamente es un apoyo a tu texto.
Así que explica lo que se ve en la Figura, haciendo referencia a la misma tal y como ves aquí.
Por ejemplo: En la Figura~\ref{fig:arquitectura} se puede ver que la estructura del \emph{parser} básico, que consta de seis componentes diferentes: los datos se obtienen de la red, y según el tipo de dato, se pasará a un \emph{parser} específico y bla, bla, bla\ldots

Si utilizas una base de datos, no te olvides de incluir también un diagrama de entidad-relación.


%-------------------------------------------------------------------------------
% EXPERIMENTOS Y VALIDACIÓN %
%-------------------------------------------------------------------------------

\cleardoublepage
\chapter{Experimentos y validación}
\label{chap:experimentos}

\textbf{Atención}: Este capítulo se introdujo como requisito en 2019.

Describe los experimentos y casos de test que tuviste que implementar para validar tus resultados. 
Incluye también los resultados de validación que permiten afirmar que tus resultados son correctos.

\section{Incorporación de código en la memoria}

Es bastante habitual que se reproduzcan fragmentos de código en la memoria de un TFG/TFM.
Esto permite explicar detalladamente partes del desarrollo que se ha realizado que se consideren
de especial interés. No obstante, tampoco es conveniente pasarse e incluir demasiado código en
la memoria, puesto que se puede alargar mucho el documento. Un recurso muy habitual es subir
todo el código a un repositorio de un servicio de control de versiones como GitHub o GitLab,
y luego incluir en la memoria la URL que enlace a dicho repositorio.

Para incluir fragmentos de código en un documento \LaTeX se pueden combinar varias
herramientas:

\begin{itemize}
    \item El entorno \mintinline{latex}{\begin{listing}[]...\end{listing}} permite crear
    un marco en el que situar el fragmento de código (parecido al generado cuando insertamos
    una tabla o una figura). Podemos insertar también una descripción (\textit{caption})
    y una etiqueta para referenciarlo luego en el texto.
    
    \item Dentro de este entorno, se puede utilizar el paquete 
    \mintinline{latex}{minted}~\footnote{\url{https://es.overleaf.com/learn/latex/Code_Highlighting_with_minted}},
    que utiliza el paquete Python Pygments para resaltado de sintaxis (coloreando el
    código). Como se puede ver en el siguiente ejemplo, hay muchas opciones de configuración
    que permiten controlar cómo se va a mostrar el código (incluir números de línea, saltos
    de línea, tamaño y tipo de fuente, espaciado, código de colores para resaltado, etc.).
\end{itemize}

\begin{listing}[h!]
    \caption{Lectura de un fichero *.csv y tipado de datos.}{}
    \label{lst:1}
    \begin{minted}[breaklines, fontsize=\footnotesize, baselinestretch=1]{python}
        # A dictionary is built to define the data type contained by each column
        dtype_scheme ={'budget': np.int64, 'genres': np.object, 'homepage': np.str, 'id': np.int64, 'keywords': np.object, 'original_language': np.str, 'original_title': np.str, 'overview': np.str, 'popularity': np.float64, 'production_companies': np.object, 'production_countries': np.object, 'release_date': np.object, 'revenue': np.int64, 'runtime': np.float64, 'spoken_languages': np.object,  'status': np.object, 'tagline': np.str, 'title': np.str, 'vote_average': np.float64, 'vote_count': np.int64}
        
        # When loading the data from the .csv file, we provide the scheme to be followed for data typing
        df1 = dd.read_csv('tmdb_5000_movies.csv', dtype=dtype_scheme)
    \end{minted}
\end{listing}

Otra ventaja del entorno \verb|listing| es que se puede generar automáticamente un índice
(con entradas hiperenlazadas) de fragmentos de código, para incluirlo al comienzo del 
documento junto con los índices de figuras, tablas, etc.

\subsection{Fuentes monoespaciadas}

A veces se incluyen nombres de archivos, paquetes, etc. como texto monoespaciado, utilizando
el comando \LaTeX \mintinline{latex}{\texttt{}}. Sin embargo, esto puede generar un problema
cuando las palabras en fuente monoespaciada alcanzan el final de una línea. En ese caso,
el compilador rehusa muchas veces romper la palabra y deja la línea demasiado larga respecto
al resto.

Para evitar esto, especialmente en párrafos más cortos de lo habitual (como en una lista
no numerada), se puede utilizar el entorno:

\begin{minted}{latex}
    \begin{sloppypar}
        ...
    \end{sloppypar}
\end{minted}

Se muestra a continuación con un ejemplo real.

\begin{itemize}
    
    \begin{sloppypar} % Arregla longitud de línea en párrafos con fuente monoespaciada
        \item Los valores contenidos en las columnas \texttt{genres}, \texttt{spoken\_languages}, \texttt{production\_companies} y \texttt{production\_countries}, clasificados originalmente como \texttt{np.objects}, se corresponden en realidad con listas de objetos \gls{json} que han sido almacenadas como cadenas de caracteres. A través de la función \texttt{get\_values(obj, key)} definida específicamente para ello, se transformará dicha cadena de caracteres en una lista de diccionarios a través de la función \texttt{json.loads(obj)} y se devolverá una  tupla que recopile los valores de los mismos para la clave indicada, un objeto de Python mucho más manejable de cara a realizar consultas sobre el \textit{dataset}.
    \end{sloppypar}
    
\end{itemize}


%-------------------------------------------------------------------------------
% CONCLUSIONES %
%-------------------------------------------------------------------------------

\cleardoublepage
\chapter{Conclusiones y trabajos futuros}
\label{chap:conclusiones}


\section{Consecución de objetivos}
\label{sec:consecucion-objetivos}

Esta sección es la sección espejo de las dos primeras del capítulo de objetivos, donde se planteaba el objetivo general y se elaboraban los específicos.

Es aquí donde hay que debatir qué se ha conseguido y qué no. 
Cuando algo no se ha conseguido, se ha de justificar, en términos de qué problemas se han encontrado y qué medidas se han tomado para mitigar esos problemas.

Y si has llegado hasta aquí, siempre es bueno pasarle el corrector ortográfico, que las erratas quedan fatal en la memoria final.
Para eso, en Linux tenemos aspell, que se ejecuta de la siguiente manera desde la línea de \emph{shell}:

\begin{minted}{bash}
    aspell --lang=es_ES -c memoria.tex
\end{minted}

\section{Aplicación de lo aprendido}
\label{sec:aplicacion}

Aquí viene lo que has aprendido durante el Grado/Máster y que has aplicado en el TFG/TFM. Una buena idea es poner las asignaturas más relacionadas y comentar en un párrafo los conocimientos y habilidades puestos en práctica.

\begin{enumerate}
    \item a
    \item b
\end{enumerate}


\section{Lecciones aprendidas}
\label{sec:lecciones_aprendidas}

Aquí viene lo que has aprendido en el Trabajo Fin de Grado/Máster.

\begin{enumerate}
    \item Aquí viene uno.
    \item Aquí viene otro.
\end{enumerate}


\section{Trabajos futuros}
\label{sec:trabajos_futuros}

Ningún proyecto ni software se termina, así que aquí vienen ideas y funcionalidades que estaría bien tener implementadas en el futuro.

Es un apartado que sirve para dar ideas de cara a futuros TFGs/TFMs.

%-------------------------------------------------------------------------------
% APÉNDICE(S) %
%-------------------------------------------------------------------------------

\cleardoublepage
\appendix

\chapter{Contenido adicional}
\label{app:additional}

\section{Dimensiones de página}

A continuación se incluye un resumen de las dimensiones calculadas para
el diseño de maquetación de las páginas de la memoria.

\printinunitsof{mm}{\pagevalues}

\verb|\marginparwidth|: \printinunitsof{mm}\prntlen{\marginparwidth}

\verb|\marginparwidth|: \printinunitsof{pt}\prntlen{\marginparwidth}

\pagediagram

\blindtext

%-------------------------------------------------------------------------------
% KOMA-Script organization
% Backmatter: glossary, bibliography, indexes, etc.
\backmatter
%-------------------------------------------------------------------------------

%-------------------------------------------------------------------------------
% GLOSARIO(S) %
%-------------------------------------------------------------------------------

\printglossary[type=\acronymtype]

\printglossary

%-------------------------------------------------------------------------------
% BIBLIOGRAFIA %
%-------------------------------------------------------------------------------

\cleardoublepage

% https://www.overleaf.com/learn/latex/Bibliography_management_with_biblatex
\raggedright\printbibliography[heading=bibintoc,title={Referencias}]

\end{document}
