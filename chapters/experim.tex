\chapter{Experimentos y validación}
\label{chap:experimentos}

\textbf{Atención}: Este capítulo se introdujo como requisito en 2019.

Describe los experimentos y casos de test que tuviste que implementar para validar tus resultados. 
Incluye también los resultados de validación que permiten afirmar que tus resultados son correctos.

\section{Incorporación de código en la memoria}

Es bastante habitual que se reproduzcan fragmentos de código en la memoria de un TFG/TFM.
Esto permite explicar detalladamente partes del desarrollo que se ha realizado que se consideren
de especial interés. No obstante, tampoco es conveniente pasarse e incluir demasiado código en
la memoria, puesto que se puede alargar mucho el documento. Un recurso muy habitual es subir
todo el código a un repositorio de un servicio de control de versiones como GitHub o GitLab,
y luego incluir en la memoria la URL que enlace a dicho repositorio.

Para incluir fragmentos de código en un documento \LaTeX se pueden combinar varias
herramientas:

\begin{itemize}
    \item El entorno \mintinline{latex}{\begin{listing}[]...\end{listing}} permite crear
    un marco en el que situar el fragmento de código (parecido al generado cuando insertamos
    una tabla o una figura). Podemos insertar también una descripción (\textit{caption})
    y una etiqueta para referenciarlo luego en el texto.
    
    \item Dentro de este entorno, se puede utilizar el paquete 
    \mintinline{latex}{minted}~\footnote{\url{https://es.overleaf.com/learn/latex/Code_Highlighting_with_minted}},
    que utiliza el paquete Python Pygments para resaltado de sintaxis (coloreando el
    código). Como se puede ver en el siguiente ejemplo, hay muchas opciones de configuración
    que permiten controlar cómo se va a mostrar el código (incluir números de línea, saltos
    de línea, tamaño y tipo de fuente, espaciado, código de colores para resaltado, etc.).
\end{itemize}

\begin{listing}[h!]
    \caption{Lectura de un fichero *.csv y tipado de datos.}{}
    \label{lst:1}
    \begin{minted}[breaklines, fontsize=\footnotesize, baselinestretch=1]{python}
        # A dictionary is built to define the data type contained by each column
        dtype_scheme ={'budget': np.int64, 'genres': np.object, 'homepage': np.str, 'id': np.int64, 'keywords': np.object, 'original_language': np.str, 'original_title': np.str, 'overview': np.str, 'popularity': np.float64, 'production_companies': np.object, 'production_countries': np.object, 'release_date': np.object, 'revenue': np.int64, 'runtime': np.float64, 'spoken_languages': np.object,  'status': np.object, 'tagline': np.str, 'title': np.str, 'vote_average': np.float64, 'vote_count': np.int64}
        
        # When loading the data from the .csv file, we provide the scheme to be followed for data typing
        df1 = dd.read_csv('tmdb_5000_movies.csv', dtype=dtype_scheme)
    \end{minted}
\end{listing}

Otra ventaja del entorno \verb|listing| es que se puede generar automáticamente un índice
(con entradas hiperenlazadas) de fragmentos de código, para incluirlo al comienzo del 
documento junto con los índices de figuras, tablas, etc.

\subsection{Fuentes monoespaciadas}

A veces se incluyen nombres de archivos, paquetes, etc. como texto monoespaciado, utilizando
el comando \LaTeX \mintinline{latex}{\texttt{}}. Sin embargo, esto puede generar un problema
cuando las palabras en fuente monoespaciada alcanzan el final de una línea. En ese caso,
el compilador rehusa muchas veces romper la palabra y deja la línea demasiado larga respecto
al resto.

Para evitar esto, especialmente en párrafos más cortos de lo habitual (como en una lista
no numerada), se puede utilizar el entorno:

\begin{minted}{latex}
    \begin{sloppypar}
        ...
    \end{sloppypar}
\end{minted}

Se muestra a continuación con un ejemplo real.

\begin{itemize}
    
    \begin{sloppypar} % Arregla longitud de línea en párrafos con fuente monoespaciada
        \item Los valores contenidos en las columnas \texttt{genres}, \texttt{spoken\_languages}, \texttt{production\_companies} y \texttt{production\_countries}, clasificados originalmente como \texttt{np.objects}, se corresponden en realidad con listas de objetos \gls{json} que han sido almacenadas como cadenas de caracteres. A través de la función \texttt{get\_values(obj, key)} definida específicamente para ello, se transformará dicha cadena de caracteres en una lista de diccionarios a través de la función \texttt{json.loads(obj)} y se devolverá una  tupla que recopile los valores de los mismos para la clave indicada, un objeto de Python mucho más manejable de cara a realizar consultas sobre el \textit{dataset}.
    \end{sloppypar}
    
\end{itemize}

\section{Conteniodo matemático y ecuaciones}
\label{sec:equations}

El contenido matemático se puede escribir en línea utilizando los delimitadores
\mintinline{latex}|\(...\)|, o bien delimitado por dos símbolos de \$. Actualmente, se prefiere
la primera opción para favorecer la compatibilidad con otras herramientas como Pandoc, así como
para aprovechar el código para escribir ecuaciones con motores de interpretación de contenido
matemático en HTML.

Si queremos que el contenido matemático se muestra en un bloque aparte, con una notación que ocupe
más espacio vertical (para sumatorios, integrales, fracciones, etc.), entonces debemos usar el llamado
\textit{display mode}, delimitado por \mintinline{latex}|\[...\]|, o bien por dos símbolos \$\$. 
De nuevo, la primera opción es preferible.

Por ejemplo, para mostrar en la misma línea una ecuación de segundo grado, escribimos \(3x^2 + 2x + 1 = 0\).
Si preferimos el contenido en un bloque independiente:

\[
F(x)=\int^a_b \frac{1}{3}x^3
\]

\[\frac{1}{\sqrt{x}}\]

\[\left(
\begin{matrix}
    1 & 0\\
    0 & 1
\end{matrix}
\right)\]

Ahora bien, en un documento en el que tenemos que hacer referencia a las ecuaciones dentro del
texto de explicación, es conveniente otorgarle a cada bloque matemático un identificador numérico, similar
al que usamos en figuras y tablas. Para ello, empleamos el bloque \mintinline{latex}|\begin{equation} contenidos... \end{equation}|.
En esta ocasión ya no tenemos que incluir los delimitadores de contenido matemático en \textit{display mode}.

\begin{equation}
    \label{eq:fraction}
    \frac{1}{\sqrt{x}}
\end{equation}

Si añadimos una etiqueta a la ecuación, entonces podemos referenciar la Ecuación \ref{eq:fraction} en el texto.
Veamos más ejemplos con caracteres especiales.

%% Testing math equations with tildes, hats, etc.

\begin{equation}
    \hat f(\omega)=\frac{1}{2\pi}
    \sum_{-\infty}^\infty f(x)e^{-i\omega x}\,dx\,, 
\end{equation}

\begin{equation}
    \tilde f(\omega)=\frac{1}{2\pi}
    \int_{-\infty}^\infty f(x)e^{-i\omega x}\,dx\,, 
\end{equation}

\begin{equation}
    \dot{\vec \omega}=\vec r\times\vec I\,.
\end{equation}


\subsection{Ecuaciones en múltiples líneas y elementos alineados}

Algunos ejemplos que muestran alineamiento de ecuaciones en diferentes líneas con 
el entorno \mintinline{latex}|\begin{align*} ... \end{align*}|.

\begin{center}
    \begin{align*}
        x&=y           &  w &=z              &  a&=b+c\\
        2x&=-y         &  3w&=\frac{1}{2}z   &  a&=b\\
        -4 + 5x&=2+y   &  w+2&=-1+w          &  ab&=cb
    \end{align*}
\end{center}

\textbf{Nota}: las ecuaciones anteriores en varias líneas tienen un modo de escritura matemática definido
para cada línea, no globalmente (a nivel de ecuación).

\begin{align*}
    f(x) &= x^2\\
    g(x) &= \frac{1}{x}\\
    F(x) &= \int^a_b \frac{1}{3}x^3
\end{align*}

\subsection{Teoremas y conjuntos}

\begin{theorem}
    Para todo número entero no negativo $n$, tenemos:
    \[ (1+x)^n = \sum_{i=0}^n {n \choose i} x^i. \]
\end{theorem}

La expansión en series de Taylor de la función $e^x$ viene dada por:

\begin{equation}
    e^x = 1 + x + \frac{x^2}{2} + \frac{x^3}{6} + \cdots = \sum_{n\geq 0} \frac{x^n}{n!}\,.
\end{equation}

\[ \forall x \in X, \quad \exists y \leq \epsilon\,, \]

\[ \frac{n!}{k!(n-k)!} = \binom{n}{k}\,. \]

\begin{theorem}
    Para cualesquiera conjuntos $A$, $B$ y $C$, se cumple que:
    \[ (A\cup B)-(C-A) = A \cup (B-C)\,. \]
\end{theorem} 
