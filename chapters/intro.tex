\chapter{Introducción}
\label{sec:intro}
\pagenumbering{arabic} % para empezar la numeración de página con números

En este capítulo se introduce el proyecto.
Debería tener información general sobre el mismo, dando la información sobre el contexto en el que se ha desarrollado.

No te olvides de echarle un ojo a la página con los cinco errores de escritura más frecuentes\footnote{\url{http://www.tallerdeescritores.com/errores-de-escritura-frecuentes}}.

Aconsejo a todo el mundo que mire y se inspire en memorias pasadas.
Las memorias de los proyectos que he llevado yo están (casi) todas almacenadas en mi web del GSyC\footnote{\url{https://gsyc.urjc.es/~grex/pfcs/}}.

\section{Sección}
\label{sec:seccion}

Esto es una sección, que es una estructura menor que un capítulo. 

Por cierto, a veces me comentáis que no os compila por las tildes.
Eso es un problema de codificación.
Al guardar el archivo, guardad la codificación de ``ISO-Latin-1'' a ``UTF-8'' (o viceversa) y funcionará.

\subsection{Estilo}
\label{subsec:estilo}

Recomiendo leer los consejos prácticos sobre escribir documentos científicos en \LaTeX \ de Diomidis Spinellis\footnote{\url{https://github.com/dspinellis/latex-advice}}.

Lee sobre el uso de las comas\footnote{\url{http://narrativabreve.com/2015/02/opiniones-de-un-corrector-de-estilo-11-recetas-para-escribir-correctamente-la-coma.html}}. 
Las comas en español no se ponen al tuntún.
Y nunca, nunca entre el sujeto y el predicado (p.ej. en ``Yo, hago el TFG'' sobre la coma).
La coma no debe separar el sujeto del predicado en una oración, pues se cortaría la secuencia natural del discurso.
No se considera apropiado el uso de la llamada coma respiratoria o \emph{coma criminal}.
Solamente se suele escribir una coma para marcar el lugar que queda cuando omitimos el verbo de una oración, pero es un caso que se da de manera muy infrecuente al escribir un texto científico (p.ej. ``El Real Madrid, campeón de Europa'').

A continuación, viene una figura, la Figura~\ref{figura:foro_hilos}. 
Observarás que el texto dentro de la referencia es el identificador de la figura (que se corresponden con el ``label'' dentro de la misma). 
También habrás tomado nota de cómo se ponen las ``comillas dobles'' para que se muestren correctamente. 
Nota que hay unas comillas de inicio (``) y otras de cierre (''), y que son diferentes.
Volviendo a las referencias, nota que al compilar, la primera vez se crea un diccionario con las referencias, y en la segunda compilación se ``rellenan'' estas referencias. 
Por eso hay que compilar dos veces tu memoria.
Si no, no se crearán las referencias.

\begin{figure}
    \centering
    \includegraphics[bb=0 0 800 600, width=12cm, keepaspectratio]{img/foro1}
    \caption{Página con enlaces a hilos}
    \label{figura:foro_hilos}
\end{figure}

A continuación un bloque ``verbatim'', que se utiliza para mostrar texto tal cual.
Se puede utilizar para ofrecer el contenido de correos electrónicos, código, entre otras cosas.

{\footnotesize
    \begin{verbatim}
        From gaurav at gold-solutions.co.uk  Fri Jan 14 14:51:11 2005
        From: gaurav at gold-solutions.co.uk (gaurav_gold)
        Date: Fri Jan 14 19:25:51 2005
        Subject: [Mailman-Users] mailman issues
        Message-ID: <003c01c4fa40$1d99b4c0$94592252@gaurav7klgnyif>
        Dear Sir/Madam,
        How can people reply to the mailing list?  How do i turn off
        this feature? How can i also enable a feature where if someone
        replies the newsletter the email gets deleted?
        Thanks
        From msapiro at value.net  Fri Jan 14 19:48:51 2005
        From: msapiro at value.net (Mark Sapiro)
        Date: Fri Jan 14 19:49:04 2005
        Subject: [Mailman-Users] mailman issues
        In-Reply-To: <003c01c4fa40$1d99b4c0$94592252@gaurav7klgnyif>
        Message-ID: <PC173020050114104851057801b04d55@msapiro>
        gaurav_gold wrote:
        >How can people reply to the mailing list?  How do i turn off
        this feature? How can i also enable a feature where if someone
        replies the newsletter the email gets deleted?
        See the FAQ
        >Mailman FAQ: http://www.python.org/cgi-bin/faqw-mm.py
        article 3.11
    \end{verbatim}
}

%%-- Objetivos del  proyecto
%%-- Si la sección anterior ha quedado muy extensa, se puede considerar convertir
%%-- Las siguientes tres secciones en un capítulo independiente de la memoria

\section{Objetivos del proyecto}
\label{sec:objetivos}

\subsection{Objetivo general} % título de subsección (se muestra)
\label{sec:objetivo-general} % identificador de subsección (no se muestra, es para poder referenciarla)


Aquí vendría el objetivo general en una frase:
Mi Trabajo Fin de Grado/Master consiste en crear de una herramienta de análisis de los comentarios jocosos en repositorios de software libre alojados en la plataforma GitHub.

Recuerda que los objetivos siempre vienen en infinitivo.


\subsection{Objetivos específicos}
\label{sec:objetivos-especificos}

Los objetivos específicos se pueden entender como las tareas en las que se ha desglosado el objetivo general. Y, sí, también vienen en infinitivo.

Lo mejor suele ser utilizar una lista no numerada, como sigue:

\begin{itemize}
    \item Un objetivo específico.
    \item Otro objetivo específico.
    \item Tercer objetivo específico.
    \item \ldots
\end{itemize}

\section{Planificación temporal}
\label{sec:planificacion-temporal}

Es conveniente que incluyas una descripción de lo que te ha llevado realizar el trabajo.
Hay gente que añade un diagrama de GANTT.
Lo importante es que quede claro cuánto tiempo has consumido en realizar el TFG/TFM 
(tiempo natural, p.ej., 6 meses) y a qué nivel de esfuerzo (p.ej., principalmente los 
fines de semana).

\section{Estructura de la memoria}
\label{sec:estructura}

Por último, en esta sección se introduce a alto nivel la organización del resto del documento
y qué contenidos se van a encontrar en cada capítulo.

\begin{itemize}
    \item En el primer capítulo se hace una breve introducción al proyecto, se describen los objetivos del mismo y se refleja la planificación temporal.
    \item En el siguiente capítulo se describen las tecnologías utilizadas en el desarrollo de este TFM/TFG (Capítulo~\ref{chap:tecnologias}).
    \item En el capítulo~\ref{chap:diseño} Se describe el proceso de desarrollo
    de la herramienta \ldots
    \item En el capítulo~\ref{chap:experimentos} Se presentan las principales pruebas realizadas
    para validación de la plataforma/herramienta\ldots (o resultados de los experimentos
    efectuados).
    \item Por último, se presentan las conclusiones del proyecto así como los trabajos futuros que podrían derivarse de éste (Capítulo~\ref{chap:conclusiones}).
\end{itemize}
